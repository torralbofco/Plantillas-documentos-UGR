% !TEX program = pdflatex
% !TEX encoding = UTF-8 Unicode

% Plantilla examen UGR
% Francisco Torralbo Torralbo, ftorralbo@ugr.es

\documentclass[monocromo]{UGR-examen}

% -------------------------------------------------------------------
% INFORMACIÓN DEL DOCUMENTO
% -------------------------------------------------------------------

% Datos del examen
\newcommand{\exGrado}{Grado\xspace}
\newcommand{\exDepartamento}{Departamento\xspace}
\newcommand{\exAsignatura}{Asignatura\xspace}
\newcommand{\exCursoAcademico}{Curso académico 23/24\xspace}
\newcommand{\exConvocatoria}{ordinario\xspace}
\newcommand{\exFecha}{17 de julio de 2024\xspace}
\newcommand{\exInstrucciones}{Todas las respuestas han de estar razonadas. Todos los ejercicios tienen la misma puntuación. Las hojas deben entregarse numeradas y llevar todas el nombre y grupo, situando esta hoja la primera.\xspace}

\begin{document}

% Bloque cabecera examen
\vspace*{-2em}
\begin{center}
  \itshape
  {\large Examen \exConvocatoria. \exFecha}\\
  \exAsignatura. \exGrado.
\end{center}
\vspace{-1em}   
\normalsize{Apellidos, Nombre}: \hrulefill\hrulefill\hrulefill\hrulefill Grupo: \hrulefill 
% --------------------------------


% Texto del examen
\begin{enumerate}
  \item Primer ejercicio
  \item \textbf{[2 puntos]} Segundo ejercicio
    \begin{enumerate}
      \item Primera cuestión
      \item Segunda cuestión
    \end{enumerate}
\end{enumerate}


\vfill

{
  \footnotesize
  \exInstrucciones
}


% Si es necesario que el documento tenga varias páginas descomentar el siguiente bloque:

% \newpage
% \thispagestyle{empty}
%
% Texto en nueva página.

\end{document}
