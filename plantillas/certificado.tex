% !TEX program = pdflatex
% !TEX encoding = UTF-8 Unicode

% Plantilla documento genérico UGR
% Francisco Torralbo Torralbo, ftorralbo@ugr.es
% 2023-02-17

\documentclass{UGR-generico}

% -------------------------------------------------------------------
% INFORMACIÓN DEL DOCUMENTO
% -------------------------------------------------------------------

\Titulo{Título}
\Nombre{Nombre Apellidos}
\Telefono{958 000 000}
\Email{profesor@ugr.es}
\URL{\url{www.ugr.es/~profesor}}
\Cargo{Cargo}
\Departamento{Departamento}
\Centro{Facultad o escuela}
\Direccion{Fuentenueva s/n, 18071 Granada}
\Universidad{Universidad de Granada}

\begin{document}

\hfill\vfill
\begin{bfseries}
D./Dña. \printNombre, \printCargo, \printDepartamento 
\end{bfseries}
\bigskip

\begin{center}
  \bfseries\scshape\large
  Certifica
\end{center}
\bigskip

Que, según los antecedentes que obran en [nombre del órgano], [NOMBRE Y APELLIDOS DEL INTERESADO/A], con DNI/Pasaporte [número con letra], Lorem ipsum dolor sit amet, consectetuer adipiscing elit. Maecenas porttitor congue massa. Fusce posuere, magna sed pulvinar ultricies, purus lectus malesuada libero, sit amet commodo magna eros quis urna.
Nunc viverra imperdiet enim. Fusce est. Vivamus a tellus.

Pellentesque habitant morbi tristique senectus et netus et malesuada fames ac turpis egestas. Proin pharetra nonummy pede. Mauris et orci.

Y para que conste y surta efectos donde convenga, expido el presente certificado, a petición del/de la interesada, en Granada, a \today. 

% Para firmar el documento digitalmente, eliminar el siguiente bloque te texto.
\vfill\vfill
\begin{center}

  \includegraphics[width=7cm]{img/firma}

  \begin{bfseries}
  \printNombre\\
  \printCargo
  \end{bfseries}
\end{center}
\vfill


% Si es necesario que el documento tenga varias páginas descomentar el siguiente bloque:

% \newpage
% \thispagestyle{empty}
%
% Texto en nueva página.

\end{document}
